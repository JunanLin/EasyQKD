\documentclass{article}
\usepackage[letterpaper, total={6in, 9in}]{geometry}
\usepackage{amssymb,mathtools}
\usepackage{amsmath,dsfont,physics}
\usepackage{mathrsfs}
\usepackage{graphicx}
\usepackage{dcolumn}
\usepackage{hyperref,verbatim} 
\usepackage{float}
\usepackage{color}
\usepackage[utf8]{inputenc}
\usepackage[english]{babel}
\usepackage{amsthm}
\usepackage{ulem}
\usepackage{appendix}
\usepackage{slashbox}
\usepackage{multirow,hhline}
\usepackage[table]{xcolor}
\usepackage{cleveref}




\newcommand{\dket}[1]{\lvert #1 \rangle\!\rangle}
\newcommand{\dbra}[1]{\langle\!\langle #1 \rvert}
\newcommand{\dbraket}[2]{\langle\!\langle #1 \vert #2 \rangle\!\rangle}



\def \todo #1{\textcolor{red}{#1}}


\begin{document}

\title{Quantum candies and quantum cryptography}
\author{Kayla Jacobs, Junan Lin, Tal Mor}
\date{}

\maketitle


In this work, we provide a pedagogical model to explain various quantum cryptography protocols that is easy to understand, and has exact correspondence with the real protocols.
%Our model can be understood by someone with zero knowledge about quantum mechanics, or even about the number $\sqrt{2}$.




\section{Introduction}
Quantum cryptography is arguably one of the most important applications of quantum information science.
Nowadays there exists many proposals on how information can be securely transferred, whose safety are based on the laws of quantum mechanics~\cite{bennett1984quantum}\todo{(add more citations)}.
On the other hand, the theory behind these protocols may be abstruse for someone without a background in quantum physics and quantum information.
This hinders general audiences from correctly understanding these concepts and is unhelpful for publicizing quantum technologies.

We provide a simple but powerful model, which contains nothing more than a candy-producing machine and some candies, that is capable of explaining various quantum cryptography protocols.
In general, we expect out model to be fully understandable by a layperson, presumably someone who has no knowledge about complex numbers or even the number $\sqrt{2}$.
On the other hand, unlike some previous constructions, our model has an exact correspondence with the true quantum mechanical model.
This makes it a powerful tool for understanding and developing more complex protocols.

\section{Quantum candy model}
Imagine that there exists a machine which produces candies.
Each candy can have multiple \textit{properties}, examples of which include color, taste, texture, etc.
Under each property there may be multiple \textit{characters}: for example, the color of a candy can be red or green or blue, the taste can be chocolate or caramel or milk.
Different properties can coexist on a single candy: for example, a candy can have a red color, a chocolate taste, and a crunchy texture.
However, a candy cannot show more than one character at the same time for each property.
For example, it cannot be both red and green.
Each candy produced by the machine will be wrapped in an identical piece of edible but opaque candy paper, such that none of its properties can be known by only looking at its appearance.

When a person uses the machine produces a candy, he/she must specify a character for \textit{one} property, and one property only: other properties that do not get specified will be randomly assigned a character by the machine, which is unknown to the outside world.
The machine does not allow specifying more than one property.
For example, Alice can input to the machine that she would like to prepare a red candy.
She knows for certain that she will get a candy with red color, but is completely ignorant about its other properties, such as its taste.
Here we must emphasize that there is no way to get around this ignorance: it is absolutely impossible to, for example, hack the machine and obtain this additional information.
It is then natural to ask what can we learn from a candy that has been produced.
With non-quantum candies, it is clear that one can know both the color and taste of a candy unambiguously; however, this is forbidden for quantum candies.
We require that depending on what was done to the candy, only the exact character corresponding to \textit{one} property can be known.
If a person looks at the candy, he/she will know exactly what color it has, but then cannot know its taste or any other property.
Note that while this appears to be a strange requirement for a candy, it \textit{can} be enforced using a classical rule.
Consider a candy with two properties, color and taste, for example.
We may setup the candy machine such that if the candy is unwrapped and exposed to light, a chemical reaction will be triggered such that its content will immediately become completely tasteless.
Thus one only has two options upon receiving a candy: they can either taste and swallow it (without unwrapping), making its color unknown during the process; or unwrap it and see its color, thereby destroying its taste.
A similar model can be constructed for candies with more properties, although this becomes a less trivial task.

Let's summarize the key information presented above.
First, the candy machine prepares candies with only one definite property, and randomly assigns a character to other properties.
Second, no information about a candy can be obtained just from its appearance.
Third, depending on what was done to a candy, at most one of its property can be precisely known, and information on its other properties will be permanently lost during this process.

\section{Relating candy and quantum picture}
As their name suggests, quantum candies are designed so that they exhibit quantum-like behaviors.
The most important characteristics of the candies that allow them to demonstrate most quantum cryptography protocols, are complementarity and randomness.
Complementarity implies that one cannot simultaneously determine multiple non-commuting observables of a quantum system, while randomness means that when the system is measured, the actual outcome is randomly drawn from all possible outcomes according to some probability distribution.
In the candy model, different properties of a candy correspond to non-commuting observables in quantum mechanics, and the characters for each property correspond to possible outcomes of an observable when a measurement is made.
The machine illustrates the procedure of preparing a quantum system in the eigenstate of one observable.
The process of trying to learn a property of a candy represents a projective measurement of the quantum system in some basis.
In this process, complementarity between observables is enforced by requiring that one can only learn about one property of a candy, after which the candy is considered destroyed (in an analogy with projective measurements of some quantum systems such as photons).
Quantum randomness is represented by internally creating random (but fixed) characters within the machine for properties that are not specified, which cannot be learned from the outside.


We point out briefly here that the different properties of a candy are not only complementary to each other, but also \textit{mutually unbiased}, meaning that if it is prepared with one definite property $A$, but one tries to learn about another property $B$, then the outcomes would not be biased towards any characters of $B$, resulting in a uniform probability distribution.
This was designed to represent the concept of mutually unbiased bases (MUBs) in quantum systems.


\section{Quantum cryptography with candies}
We now demonstrate the implementation of several common quantum cryptography protocols using these candies.
In particular, many quantum key distribution (QKD) protocols can be demonstrated easily with this model.
Key distribution is the essential component in public key cryptography, a practicable way to achieve safe classical communication.
The safety of public key cryptography relies on securely distributing the key from the sender to the receiver in order to encrypt and decrypt the message, while not leaking the key to a malice third party.
The problem being considered is as follows: Alice would like to transmit a string of random bits to Bob, so that the string can then be used as a secret key (known only to Alice and Bob) to encrypt meaningful messages.
The classical communication channel between the two parties is unsafe, and a quantum channel (which allows them to send qubits to each other) could also be eavesdropped by a third party.
Alice and Bob would like to know about the presence of an eavesdropper, provided that he/she holds some information about the secret key being sent. 

\subsection{The BB84 Protocol}

The BB84 protocol is a well-known QKD protocol proposed in 1984, whose name came from its proposers Bennett and Brassard~\cite{bennett1984quantum}.
Their scheme allows Alice and Bob to detect the presence of a potential eavesdropper who may be trying to tap on their secret bits, and we will show that they can achieve this using a machine that produces candies with two properties, say color and taste.
In addition, they also need a classical communication channel and a method to deliver candies from Alice to Bob, which can both be unsafe.

Prior to sending anything, she meets up with Bob (or communicates through the classical channel) and agree on a set of rules that assign the bit 0 to red color and caramel taste, and 1 to green color and chocolate taste.
Then, the two parties are separated and can only communicate through the classical channel and the candy channel.
Alice repeatedly does the following: she produces a random bit, and makes a candy with a random property that has a character matching her bit.
For example, if she wants to send the bit 0, she can either make a red candy, or a candy with caramel taste.
This candy is then delivered to Bob.
Upon receiving, Bob randomly chooses what he would like to do with the candy: he can either open the cover and record its color, or eat it and record its taste.
He then translates this obtained information into 0 or 1 according to the predefined rules, and record the bit in the order of receiving the candies.
After having sent a sufficient number of candies, Alice would announce to Bob through the classical channel whether she prepared a candy with a definite color or taste for all candies, in the order of preparation.
Bob then compares this list of production methods with his own method of retrieving the information, and only keep the bits under the cases where the two methods match.
This establishes a string of random bits shared between Alice and Bob.
Clearly, there are in total 4 possibilities, listed under \cref{table1}.


\begin{table}[h!]
\centering
\begin{tabular}{ |c|c|c| } 
 \hline
  \backslashbox{Alice}{Bob} & Look & Taste \\ 
  \hline 
 Color & Keep & Discard \\
 \hline 
 Taste & Discard & Keep \\ 
 \hline
\end{tabular}
\caption{Actions by Alice and Bob in candy BB84 protocol.}
\label{table1}
\end{table}

Suppose that Eve, in possess of a same candy machine as Alice, wants to tap on Alice's and Bob's secret bits.
Because Eve does not know how Alice prepared her candies, she randomly looks at or tastes a candy, in the process destroying the other property. 
The best that Eve can do is to prepare a candy with the same property as known by her, and send that to Bob.
Upon receiving from Eve, Bob (who does not know about the presence of Eve) randomly measures this new candy on its color or taste.
In the cases where Eve's preparation does not match with Alice's, there is a 50\% probability that Alice and Bob would notice a mismatch between their bits, as shown in \cref{table2}.
Overall, if Eve has intercepted a total of $N$ candies from Alice, then there will be approximately $N/4$ mismatched bits in the ones shared by Alice and Bob, which can be revealed if they compare a subset of their bits through the classical channel.
Under this case, Alice and Bob can either discard their string and resend later, or use protocols such as privacy amplification to reduce the portion  \todo{(or use some other methods mentioned in other literatures? e.g., privacy amplification?)}.

\begin{table}[h!]
	\centering
	\begin{tabular}{ |c | c|c|c|c| } 
		\hline
		\backslashbox{Bob}{Alice} & \multicolumn{2}{|c|}{Color} & \multicolumn{2}{|c|}{Taste} \\ 
		\hhline{*{5}{-}}
		\multirow{2}{5em}{Look} &  \multicolumn{2}{|c|}{ \cellcolor[HTML]{9af9a1} Look, 50\% } & \multicolumn{2}{|c|}{\multirow{2}{*}{Discard}} \\
		\cline{2-3}
		& \cellcolor[HTML]{f7bebe} Taste, 25\% & \cellcolor[HTML]{9af9a1} Taste, 25\% & \multicolumn{2}{|c|}{} \\
		\hline
		\multirow{2}{5em}{Taste} & \multicolumn{2}{|c|}{\multirow{2}{*}{Discard}} & \cellcolor[HTML]{f7bebe} Look, 25\% & \cellcolor[HTML]{9af9a1} Look, 25\% \\ 
		\hhline{| ~ | ~  ~ | - | - |}
		& \multicolumn{2}{|c|}{} & \multicolumn{2}{|c|}{ \cellcolor[HTML]{9af9a1} Taste, 50\%} \\
		\hline
	\end{tabular}
	\caption{Truth table in the presence of an eavesdropper. The table entries indicate actions taken by the eavesdropper as well as their occurring probabilities. A green cell implies agreement between Alice and Bob, whereas a red cell implies a disagreement. Under 25\% of the cases their bits do not agree, which can be revealed by comparing a sub-string that is sufficiently long.}
	\label{table2}
\end{table}

\subsection{The B92 protocol}
The B92 protocol was proposed by Bennett in 1992~\cite{bennett1992quantum}, originally as a simplified version of the BB84 protocol.
Here, Alice would use the same machine as she used in the BB84 protocol and the candies also have two properties: color and taste.
However, she only prepares two types of candies: red and chocolate, and tells Bob that they stand for bits 0 and 1 respectively.
When Bob receives a candy, he also randomly chooses to taste or look at it, but does not reveal his choice to Alice as in BB84.
Instead, he tells Alice whether he \textit{succeeds} or \textit{fails} for every candy through the classical channel, where the success condition is defined as obtaining a character that was not prepared by Alice.
In other words, Bob succeeds if he tastes caramel or sees green, and he associates green with the bit 1 and caramel with 0.
After Alice has sent enough number of bits, they keep only the succeeded cases and discard the failed ones.
As in BB84, they then compare a subset of the bits to determine whether an eavesdropper is present.

\begin{table}[h!]
	\centering
	\begin{tabular}{ |c|c|c| } 
		\hline
		\backslashbox{Alice}{Bob} & Look & Taste \\ 
		\hline 
		Red (0) & Discard & Keep if tastes Caramel \\
		\hline 
		Chocolate (1) & Keep if sees Green & Discard \\ 
		\hline
	\end{tabular}
	\caption{Actions by Bob in candy B92 protocol.}
	\label{table3}
\end{table}

It is easy to understand the reasoning behind this protocol in the candy picture.
Since Alice did not make a green candy, any such cases must come from the cases where Alice chose to prepare a chocolate candy, indicating that she was trying to send a bit 1.
The same argument applies for tasting caramel.
On the other hand, seeing red is an inconclusive result, because it can happen regardless of which bit Alice was sending.
If Eve intercepts Alice's candy, she cannot make a perfect copy and resend to Bob due to her finite probability of failure, which will affect the joint result between Alice and Bob.

\subsection{The six-state protocol}
The six-state protocol is another generalization to the original BB84 protocol, where all 3 pairs of orthonormal basis states for a qubit are used.
For this protocol, we assume that the machine is now capable of preparing candies with three properties.
As previously mentioned, we can imagine the candies as having a color, a taste, and a texture, each having 2 characters.
The texture of a candy can be, for example, soft or crunchy.
On the other hand, the requirement that the machine can only prepare a candy with one definite property does not change; so does the condition that one can only know one property of a candy.
For an analogy to qubits, the three properties correspond to three mutually unbiased bases (MUBs), saturating the upper bound for a qubit.
The protocol is otherwise identical to the BB84 protocol, where Alice randomly chooses a property and a character, generates a candy and sends to Bob, and Bob performs an action and obtains a bit~\cite{kato2016security}.


\subsection{The qutrit protocol}
QKD can also be performed with higher dimensional quantum systems, which can provide an increased tolerance level to experimental noise~\cite{bechmann2000quantum}.
We assume now that the candies have four properties: color, taste, texture, and smell.
Each property now has three characters.
As in the six-state protocol for qubits, this saturates the upper bound for the number of MUBs for a qutrit.
The procedure for Alice and Bob are identical as in BB84, only that now there are a total of 12 choices for Alice to prepare and send a candy, and 4 ways for Bob to examine it, compared with 4 choices and 2 ways in BB84.
Upon successful matching of one candy, Alice and Bob would share a trit of information.
And just like in the other protocols, if Eve examines a different property than Alice's choice, she would (statistically) inevitably introduce errors to Bob's results.

\section{Discussion}
A similar ``classical implementation'' of the BB84 protocol has been discussed before~\cite{svozil2006staging,Svozil2014}. 
The difference is that while in~\cite{svozil2006staging} the complementary observables corresponding to the two basis are demonstrated using different color filters, in our model the complementarity is enforced by the properties of the chocolate candies themselves.
This removes the requirement in~\cite{svozil2006staging} that Eve must also put on color glasses, so is identical to the original photonic construction.
Moreover, it was noted in~\cite{Svozil2014} that for the previous construction Alice and Bob cannot detect the presence of Eve because her measurement does not change the state of a candy.
In comparison, we have provided a construct where complementarity can be understood under a very classical setting.
Of course, as also noted in~\cite{Svozil2014}, the BB84 protocol did not make use of contextuality~\cite{kochen1975problem,bell2001problem}, another crucial resource of quantum systems~\cite{howard2014contextuality}.
This makes it possible to construct ``classical'' analogies which are less ``counterintuitive'' and do not, for example, violate a general Bell inequality.

The picture with classical complementary observables of chocolate candies can also be helpful in discussions about nonlocality.
For example, non-local boxes (NLBs)~\cite{popescu1994quantum} are imaginary devices that allow two players Alice and Bob to win the CHSH game~\cite{buhrman2001quantum} with probability unity, whereas the best quantum strategy has a winning probability of $\cos[2](\pi/8) \approx 0.85$~\cite{broadbent2006power}.
NLBs exhibit correlations that cannot be produced only by local rules, hence the name.
In the candy picture, the correlations of an NLB can be translated as follows.
Suppose now that the machine has been upgraded and can produce two non-local chocolate candies every time.
Recall, that we have made the bit assignment that $R = 0,\ G = 1$ and $M = 0,\ C = 1$ to the characters of a candy.
The pair of candies are produced and distributed to Alice and Bob who may be far apart and are not allowed to communicate.
Upon receiving a candy, Alice and Bob can choose to either look or taste, and will obtain a bit corresponding to the observed character.
The candies are correlated in the following way: if both Alice and Bob choose to taste, they will obtain results that correspond to the same bit, in this case the same taste.
In all other cases, they will obtain results that correspond to opposite bits.
As an example, if Alice tastes chocolate ($C = 1$), then if Bob choose to taste then he will taste chocolate ($C = 1$) too; and if Bob instead choose to look, he will observe red ($R = 0$).

Our candy model creates new possibilities to explore beyond-quantum correlations like the NLBs, but can be further generalized to include more possibilities. 
For example, the concept of mutually unbiased bases (MUBs) is an important concept in theoretical quantum information that also has practical consequences.
It has been shown that some QKD protocols using higher dimensional systems (which has a larger number of MUBs) can offer security advantages compared to the original BB84 protocol~\cite{cerf2002security}.
However, how to find the set of MUBs for a $d$--dimensional system when $d$ is not a prime power remains an unknown problem as of today~\cite{coles2017entropic}.
Using the candy model, we can easily construct systems with an arbitrary number of MUBs by assuming additional characters that the candies may have: for example, in addition to color and taste, we may assume that the candies have additional characters such as texture, smell, etc.
Their behavior with increased number of characters can provide insights onto the behavior of real quantum systems.
%it is known that the number of mutually unbiased bases (MUBs) in a $d$--dimensional Hilbert space can at most be $d+1$~\cite{bandyopadhyay2002new}.
%In the case of a qubit with $d=2$, the 3 MUBs are given by the Pauli matrices.



\bibliographystyle{naturemag}
\bibliography{references,/home/iqcmlabc/Documents/Bibliography/candyQKD}

%\clearpage
%\newpage
%\begin{appendices}
%\section{Appendix}

%\end{appendices}



\end{document}



